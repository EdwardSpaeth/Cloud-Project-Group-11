\hypertarget{section-solution-strategy}{%
\section{Solution Strategy}\label{section-solution-strategy}}

\textbf{Content}

This section provides a brief overview of the technologies used in the development of the project, along with the design patterns applied.

\subsection{Flask for the backend}
We chose Flask for the backend because it is a lightweight and flexible micro web framework, perfect for handling routing and database interactions with minimal overhead. Flask’s simplicity allows us to build and maintain our web application efficiently, which is ideal given that our project is a small-scale webshop with limited complexity. Additionally, Flask’s modular design makes it easy to extend with necessary tools or libraries without being weighed down by unnecessary components, making it a practical choice for our specific use case.

There was no need to implement a specific design pattern for the backend due to the simplicity and small scale of the application. The straightforward nature of the project allowed us to keep the backend lightweight and functional without the overhead of a complex architecture.

\subsection{Next.js for the frontend}
The main advantage of using Next.js over React is its built-in support for API routes, enabling seamless integration of backend functionality within the same framework. This allows us to easily perform CRUD operations (POST, GET, PUT, DELETE) without needing a separate server setup. Additionally, like React, Next.js empowers us to create dynamic websites with smooth user experiences, while offering features like server-side rendering and static site generation for better performance and SEO optimization.

In Next.js we decided to use the App Router Structure for the frontend due to its simplicity in establishing a clear and organized file-based routing structure. Next.js automatically handles the routing by serving the requested files based on the folder structure, which reduces the need for manual routing configuration. This makes development more efficient and ensures that our project remains maintainable as it scales.