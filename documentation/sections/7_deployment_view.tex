\hypertarget{section-deployment-view}{%
\section{Deployment View}\label{section-deployment-view}}

\textbf{Content}

The deployment view describes:

\begin{enumerate}
\def\labelenumi{\arabic{enumi}.}
\item
  technical infrastructure used to execute your system, with
  infrastructure elements like geographical locations, environments,
  computers, processors, channels and net topologies as well as other
  infrastructure elements and
\item
  mapping of (software) building blocks to that infrastructure elements.
\end{enumerate}

Often systems are executed in different environments, e.g. development
environment, test environment, production environment. In such cases you
should document all relevant environments.

Especially document a deployment view if your software is executed as
distributed system with more than one computer, processor, server or
container or when you design and construct your own hardware processors
and chips.

From a software perspective it is sufficient to capture only those
elements of an infrastructure that are needed to show a deployment of
your building blocks. Hardware architects can go beyond that and
describe an infrastructure to any level of detail they need to capture.

\textbf{Motivation}

Software does not run without hardware. This underlying infrastructure
can and will influence a system and/or some cross-cutting concepts.
Therefore, there is a need to know the infrastructure.

Maybe a highest level deployment diagram is already contained in section
3.2. as technical context with your own infrastructure as ONE black box.
In this section one can zoom into this black box using additional
deployment diagrams:

\begin{itemize}
\item
  UML offers deployment diagrams to express that view. Use it, probably
  with nested diagrams, when your infrastructure is more complex.
\item
  When your (hardware) stakeholders prefer other kinds of diagrams
  rather than a deployment diagram, let them use any kind that is able
  to show nodes and channels of the infrastructure.
\end{itemize}

See \href{https://docs.arc42.org/section-7/}{Deployment View} in the
arc42 documentation.

\hypertarget{_infrastructure_level_1}{%
\subsection{Infrastructure Level 1}\label{_infrastructure_level_1}}

Describe (usually in a combination of diagrams, tables, and text):

\begin{itemize}
\item
  distribution of a system to multiple locations, environments,
  computers, processors, .., as well as physical connections between
  them
\item
  important justifications or motivations for this deployment structure
\item
  quality and/or performance features of this infrastructure
\item
  mapping of software artifacts to elements of this infrastructure
\end{itemize}

For multiple environments or alternative deployments please copy and
adapt this section of arc42 for all relevant environments.

\emph{\textbf{\textless Overview Diagram\textgreater{}}}

\begin{description}
\item[Motivation]
\emph{\textless explanation in text form\textgreater{}}
\item[Quality and/or Performance Features]
\emph{\textless explanation in text form\textgreater{}}
\item[Mapping of Building Blocks to Infrastructure]
\emph{\textless description of the mapping\textgreater{}}
\end{description}

\hypertarget{_infrastructure_level_2}{%
\subsection{Infrastructure Level 2}\label{_infrastructure_level_2}}

Here you can include the internal structure of (some) infrastructure
elements from level 1.

Please copy the structure from level 1 for each selected element.

\hypertarget{__emphasis_infrastructure_element_1_emphasis}{%
\subsubsection{\texorpdfstring{\emph{\textless Infrastructure Element
1\textgreater{}}}{\textless Infrastructure Element 1\textgreater{}}}\label{__emphasis_infrastructure_element_1_emphasis}}

\emph{\textless diagram + explanation\textgreater{}}

\hypertarget{__emphasis_infrastructure_element_2_emphasis}{%
\subsubsection{\texorpdfstring{\emph{\textless Infrastructure Element
2\textgreater{}}}{\textless Infrastructure Element 2\textgreater{}}}\label{__emphasis_infrastructure_element_2_emphasis}}

\emph{\textless diagram + explanation\textgreater{}}

\ldots{}

\hypertarget{__emphasis_infrastructure_element_n_emphasis}{%
\subsubsection{\texorpdfstring{\emph{\textless Infrastructure Element
n\textgreater{}}}{\textless Infrastructure Element n\textgreater{}}}\label{__emphasis_infrastructure_element_n_emphasis}}

\emph{\textless diagram + explanation\textgreater{}}
