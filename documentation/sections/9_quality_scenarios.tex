\hypertarget{section-quality-scenarios}{%
\section{Quality Requirements}\label{section-quality-scenarios}}

\textbf{Content}

This section contains all quality requirements as quality tree with
scenarios. The most important ones have already been described in
section 1.2. (quality goals)

Here you can also capture quality requirements with lesser priority,
which will not create high risks when they are not fully achieved.

\textbf{Motivation}

Since quality requirements will have a lot of influence on architectural
decisions you should know for every stakeholder what is really important
to them, concrete and measurable.

See \href{https://docs.arc42.org/section-10/}{Quality Requirements} in
the arc42 documentation.

\hypertarget{_quality_tree}{%
\subsection{Quality Tree}\label{_quality_tree}}

\textbf{Content}

The quality tree (as defined in ATAM -- Architecture Tradeoff Analysis
Method) with quality/evaluation scenarios as leafs.

\textbf{Motivation}

The tree structure with priorities provides an overview for a sometimes
large number of quality requirements.

\textbf{Form}

The quality tree is a high-level overview of the quality goals and
requirements:

\begin{itemize}
\item
  tree-like refinement of the term "quality". Use "quality" or
  "usefulness" as a root
\item
  a mind map with quality categories as main branches
\end{itemize}

In any case the tree should include links to the scenarios of the
following section.

\hypertarget{_quality_scenarios}{%
\subsection{Quality Scenarios}\label{_quality_scenarios}}

\textbf{Contents}

Concretization of (sometimes vague or implicit) quality requirements
using (quality) scenarios.

These scenarios describe what should happen when a stimulus arrives at
the system.

For architects, two kinds of scenarios are important:

\begin{itemize}
\item
  Usage scenarios (also called application scenarios or use case
  scenarios) describe the system's runtime reaction to a certain
  stimulus. This also includes scenarios that describe the system's
  efficiency or performance. Example: The system reacts to a user's
  request within one second.
\item
  Change scenarios describe a modification of the system or of its
  immediate environment. Example: Additional functionality is
  implemented or requirements for a quality attribute change.
\end{itemize}

\textbf{Motivation}

Scenarios make quality requirements concrete and allow to more easily
measure or decide whether they are fulfilled.

Especially when you want to assess your architecture using methods like
ATAM you need to describe your quality goals (from section 1.2) more
precisely down to a level of scenarios that can be discussed and
evaluated.

\textbf{Form}

Tabular or free form text.
