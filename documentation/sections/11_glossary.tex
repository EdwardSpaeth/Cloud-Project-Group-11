\hypertarget{section-glossary}{%
\section{Glossary}\label{section-glossary}}

\textbf{Contents}

The most important domain and technical terms that your stakeholders use
when discussing the system.

You can also see the glossary as source for translations if you work in
multi-language teams.

\textbf{Motivation}

You should clearly define your terms, so that all stakeholders

\begin{itemize}
\item
  have an identical understanding of these terms
\item
  do not use synonyms and homonyms
\end{itemize}

A table with columns \textless Term\textgreater{} and
\textless Definition\textgreater.

Potentially more columns in case you need translations.

See \href{https://docs.arc42.org/section-12/}{Glossary} in the arc42
documentation.

\begin{longtable}[]{@{}
  >{\raggedright\arraybackslash}p{(\columnwidth - 2\tabcolsep) * \real{0.3333}}
  >{\raggedright\arraybackslash}p{(\columnwidth - 2\tabcolsep) * \real{0.6667}}@{}}
\toprule
\begin{minipage}[b]{\linewidth}\raggedright
Term
\end{minipage} & \begin{minipage}[b]{\linewidth}\raggedright
Definition
\end{minipage} \\
\midrule
\endhead
\emph{Cloud Service Provider (CSP)} &
\emph{A Cloud Service Provider (CSP) is a third-party company, 
which offers (paid) services in regard to cloud capabilities, be it compute, storage, management, and/or analytics. 
In our case, we are solely referring to public Cloud Service Providers, which provide these services to the public,
opposed to offering services exclusive to one or multiple companies.} \\ \hline
\emph{Microsoft Azure ("Azure")} &
\emph{Microsoft Azure is a public cloud service provider belonging to Microsoft.} \\ \hline
\emph{Service-Level Agreement (SLA)} &
\emph{Service-level agreements are made between a (cloud) service provider and a customer. 
For our purposes, the main point of interest, is availability, where the (cloud) service provider
guarantees a certain availability for a given service, 
allowing us to fulfill our own availability requirements when relying on said service.} \\ \hline
\emph{Horizontal Scaling} &
\emph{Horizontal Scaling refers to improving performance by spinning up multiple instances,
so that requests can be distributed among them, allowing for greater parallel processing, 
rather than just increasing hardware performance directly (Vertical Scaling).} \\ \hline
\emph{Replica} &
\emph{A replica in our case refers to additional copies of either Docker containers or database copies.
The purpose is to provide increased performance (horizontal scaling), such as having multiple read-only replicas of a database, 
allowing for higher throughput of read operations, 
especially if additional replicas are spread geographically to decrease latency.
Replicas also increases availability, especially if they are spread geographically, 
making it so that an instance of a service is running even if there were to be a data center failure at N-1 locations.} \\
\bottomrule
\end{longtable}
